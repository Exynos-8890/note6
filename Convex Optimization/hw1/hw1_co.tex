\documentclass[en,hazy,black,12pt,device = normal]{elegantnote}
\usepackage[breakable]{tcolorbox}
\usepackage{amsmath,amsthm,amssymb,bm,xcolor}
\usepackage{../codefs}
\tcbset{breakable}
\pagecolor{white}

\title{Convex Optimization Exercise week1\&2}
\author{Tianlu Zhu \\ 2020571005}
% \institute{ShanghaiTech}
\date{}
\begin{document}
\maketitle

\section*{Week 1}
\subsection*{Problem 1}
Consider the function $f: \mathbb{S}^n \rightarrow \mathbb{R}$ defined as
$$
f(X)=\log \operatorname{det} X, \quad  \dom f=\mathbb{S}_{++}^n \text {. }
$$
Compute the second order approximation of $f$.

\begin{tcolorbox}
    \sol

    \begin{align*}
        \log \det Z &= \log \det (X+ \Delta X) \\
        &= \log \det \left(X^{1/2} (I + X^{1/2} \Delta X X^{1/2} )X^{1/2}\right)\\
        &= \log \det X^{1/2} + \log \det (I + X^{1/2} \Delta X X^{1/2} ) + \log \det X^{1/2} \\
        &= \log\det X + \sumi\log (1+ \lambda _i)\\
        &\approx \log\det X + \tr(X^{-1}\Delta X)
    \end{align*}
    % Where \(\lambda _i\) is the \(i\)-th eigenvalue of \(X^{1/2} \Delta X X^{1/2} \)
    \begin{align*}
        (X+\Delta X)^{-1} & =\left(X^{1 / 2}\left(I+X^{-1 / 2} \Delta X X^{-1 / 2}\right) X^{1 / 2}\right)^{-1} \\
        & =X^{-1 / 2}\left(I+X^{-1 / 2} \Delta X X^{-1 / 2}\right)^{-1} X^{-1 / 2} \\
        & \approx X^{-1 / 2}\left(I-X^{-1 / 2} \Delta X X^{-1 / 2}\right) X^{-1 / 2} \\
        & =X^{-1}-X^{-1} \Delta X X^{-1}\\
        % \begin{aligned}
            f(Z) & =f(X+\Delta X) \\
            & \approx f(X)+\operatorname{tr}\left(X^{-1} \Delta X\right)-(1 / 2) \operatorname{tr}\left(X^{-1} \Delta X X^{-1} \Delta X\right) \\
            & \approx f(X)+\operatorname{tr}\left(X^{-1}(Z-X)\right)-(1 / 2) \operatorname{tr}\left(X^{-1}(Z-X) X^{-1}(Z-X)\right)
            % \end{aligned}
    \end{align*}
\end{tcolorbox}

\section*{Week 2}
\subsection*{Problem1 (1)}
Suppose that \(C \subset \R^n\) is convex, then \(\int C, \cl C\) are also convex.

\begin{tcolorbox}
    \sol

    Fist we consider the interior point. Suppose \(x_1,x_2 \in \int C\), \(\forall \epsilon_1,\epsilon_2\),
    \begin{align*}
        \left\{y_1 \mid\|y_1-x_1\|_2 \leq \epsilon_1\right\} \subseteq C\\
        \left\{y_2 \mid\|y_2-x_2\|_2 \leq \epsilon_2\right\} \subseteq C
    \end{align*}
    Then:
    \begin{align*}
        &\left\{y \mid\|y-\left(tx_1+(1-t) x_2\right)\|_2 \leq \epsilon\right\}\\
        =&\left\{ty_1+(1-t)y_2 \mid\|\left(t(y_1 - x_1)+(1-t) (y_2 - x_2)\right)\|_2 \leq \epsilon\right\} 
    \end{align*}
    Since \(\normt{t(y_1 - x_1)+(1-t) (y_2 - x_2)}\leq \normt{t(y_1 - x_1)} + \normt{(1-t) (y_2 - x_2)}\leq \epsilon\), one obtain that  \(tx_1+(1-t) x_2 \in\int C\). That is, \(\int C\) is also convex.
    
    Then we consider the closure. We adapt the limit definition of closure: the limit point of all \(x_n \in C\), and the proof is as the \(\int C\).
\end{tcolorbox}

\subsection*{Problem 2 (Ex 2.2)}

Show that a set is convex if and only if its intersection with any line is convex. Show that a set is affine if and only if its intersection with any line is affine.

\begin{tcolorbox}
    \sol

    For convex set, two convex set take intersection is also convex, and a line is convex, so the "only if" is trivial. Conversely, any line intersect with a set is convex, that is, every line segments between any two point are in the set, so it is convex.

    For affine set, two affine set take intersection is also convex, and a line is affine, so the "only if" is trivial. Conversely, any line intersect with a set is affine, that is, every lines between any two point are in the set, so it is affine.
\end{tcolorbox}

\subsection*{Problem 3 (Ex 2.7)}
\textit{Voronoi description of halfspace.} Let $a$ and $b$ be distinct points in $\mathbf{R}^n$. Show that the set of all points that are closer (in Euclidean norm) to $a$ than $b$, i.e., $\left\{x \mid\|x-a\|_2 \leq\|x-b\|_2\right\}$, is a halfspace. Describe it explicitly as an inequality of the form $c^T x \leq d$. Draw a picture.

\begin{tcolorbox}
    \sol
    \begin{align*}
        &\normt{x-a} \leq \normt{x-b} \\
        \Leftrightarrow\quad& (x-a)^T(x-a) \leq (x-b)^T(x-b) \\
        \Leftrightarrow\quad& x^Tx -a^Tx -x^Ta -a^Ta \leq x^Tx -b^Tx -x^Tb -b^Tb \\
        \Leftrightarrow\quad& 2(b-a)^T x \leq b^Tb-a^Ta
    \end{align*}
    So it is a halfspace.
\end{tcolorbox}

\subsection*{Problem 4 (Ex 2.18)}
\textit{Invertible linear-fractional functions.} Let $f: \mathbf{R}^n \rightarrow \mathbf{R}^n$ be the linear-fractional function
$$
f(x)=(A x+b) /\left(c^T x+d\right), \quad \operatorname{dom} f=\left\{x \mid c^T x+d>0\right\} .
$$
Suppose the matrix
$$
Q=\left[\begin{array}{cc}
A & b \\
c^T & d
\end{array}\right]
$$
is nonsingular. Show that $f$ is invertible and that $f^{-1}$ is a linear-fractional mapping. Give an explicit expression for $f^{-1}$ and its domain in terms of $A, b, c$, and $d$. \textit{Hint.} It may be easier to express $f^{-1}$ in terms of $Q$.

\begin{tcolorbox}
    \sol

    Denote
    \begin{align*}
        Q=\left[\begin{array}{cc}
            A & b \\
            c^T & d
            \end{array}\right] \in \mathbf{R}^{(m+1) \times(n+1)}
    \end{align*}
    Simply let \(f^{-1} (x) = P^{-1}\left(Q^{-1}\left(P(x)\right)\right)\), and this is the projection under \(Q^{-1}\).
\end{tcolorbox}

\end{document}