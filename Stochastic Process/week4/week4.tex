\documentclass[en,hazy,blue,12pt,device = pad]{elegantnote}
\usepackage[breakable]{tcolorbox}
\usepackage{amsmath,amsthm,amssymb,bm,xcolor}
\usepackage{../spdefs}
\tcbset{breakable}
\pagecolor{white}

\title{Stochastic Process week 3 Exercise}
\author{Tianlu Zhu \\ 2020571005}
% \institute{ShanghaiTech}
\date{}
\begin{document}
\maketitle

Chap 4: 20, 22, 23, 32, 38, 45, 47

\begin{enumerate}
    \item[20] A transition probability matrix \({\bf P}\) is said to be doubly stochastic is the sum over each column equals one; that is,
    \[\sum _i P_{ij } = 1, \quad \text{for all } j  \] 
    If such a chain is irreducible and consists of \(M+1\) states \(0,1,\cdots,M\), show that the long-run proportions are given by 
    \[\pi_j = \frac{1}{M+1},\quad j = 0,1,\cdots,M\]
    \begin{tcolorbox}
        \sol 

        
    \end{tcolorbox}
    \item[22]
    \item[23]
    \item[32] 
    \item[38] 
    \item[45] 
    \item[47]  
\end{enumerate}

\end{document}