\documentclass[en,hazy,blue,12pt,device = pad]{elegantnote}
\usepackage[breakable]{tcolorbox}
\usepackage{amsmath,amsthm,amssymb,bm,xcolor}
\usepackage{../spdefs}
\tcbset{breakable}
\pagecolor{white}

\title{Stochastic Process Exercise week 3}
\author{Tianlu Zhu \\ 2020571005}
% \institute{ShanghaiTech}
\date{}
\begin{document}
\maketitle
Chap 4: 1, 5, 13, 14, 16
\begin{enumerate}
    \item[1] Three white and three black balls are distributed in two urns in such a way that each contains three balls. We say that the system is in state \(i, i=0,1,2,3\), if the first urn contains \(i\) white balls. At each step, we draw one ball from each urn and place the ball drawn from the first urn into the second, and conversely with the ball from the second urn. Let \(X_n\) denote the state of the system after the \(n\)th step. Explain why \(\left\{X_n, n=0,1,2, \ldots\right\}\) is a Markov chain and calculate its transition probability matrix.
    
    \begin{tcolorbox}
        \sol 


    \end{tcolorbox}

    \item[5] A Markov chain \(\left\{X_n:n\geq 0 \right\}\) with states \(0,1,2\) has the transition probability matrix
    \begin{align*}
        \begin{bmatrix}
            1/2 & 1/3 & 1/6 \\
            0& 1/3 & 2/3\\
            1/2 & 0 &1/2
        \end{bmatrix}
    \end{align*}
    If \(P\{X_0 = 0\}= P(X_0 = 1) = 1/4\), find \(\mean{X_3}\).

    \begin{tcolorbox}
        \sol


    \end{tcolorbox}

    \item[13]Let \(\mathbf{P}\) be the transition probability matrix of a Markov chain. Argue that if for some positive integer \(r, \mathbf{P}^r\) has all positive entries, then so does \(\mathbf{P}^n\), for all integers \(n \geqslant r\).
    
    \begin{tcolorbox}
        \sol


    \end{tcolorbox}

    \item[14] Specify the classes of the following Markov chains, and determine whether they are transient or recurrent:
    \begin{align*}
        &\mathbf{P}_1=\left\|\begin{array}{ccc}
        0 & \frac{1}{2} & \frac{1}{2} \\
        \frac{1}{2} & 0 & \frac{1}{2} \\
        \frac{1}{2} & \frac{1}{2} & 0
        \end{array}\right\|,
        &\mathbf{P}_2=\left\|\begin{array}{llll}
        0 & 0 & 0 & 1 \\
        0 & 0 & 0 & 1 \\
        \frac{1}{2} & \frac{1}{2} & 0 & 0 \\
        0 & 0 & 1 & 0
        \end{array}\right\|,
    \end{align*}
    \begin{align*}
        &\mathbf{P}_3=\left\|\begin{array}{lllll}
        \frac{1}{2} & 0 & \frac{1}{2} & 0 & 0 \\
        \frac{1}{4} & \frac{1}{2} & \frac{1}{4} & 0 & 0 \\
        \frac{1}{2} & 0 & \frac{1}{2} & 0 & 0 \\
        0 & 0 & 0 & \frac{1}{2} & \frac{1}{2} \\
        0 & 0 & 0 & \frac{1}{2} & \frac{1}{2}
        \end{array}\right\|,
        &\mathbf{P}_4=\left\|\begin{array}{|ccccc}
        \frac{1}{4} & \frac{3}{4} & 0 & 0 & 0 \\
        \frac{1}{2} & \frac{1}{2} & 0 & 0 & 0 \\
        0 & 0 & 1 & 0 & 0 \\
        0 & 0 & \frac{1}{3} & \frac{2}{3} & 0 \\
        1 & 0 & 0 & 0 & 0
        \end{array}\right\|
    \end{align*}
    \begin{tcolorbox}
        \sol


    \end{tcolorbox}

    \item[16] Show that if state \(i\) is recurrent and state \(i\) does not communicate with state \(j\), then \(P_{i j}=0\). This implies that once a process enters a recurrent class of states it can never leave that class. For this reason, a recurrent class is often referred to as a closed class.
    
    \begin{tcolorbox}
        \sol


    \end{tcolorbox}
\end{enumerate}


\end{document}