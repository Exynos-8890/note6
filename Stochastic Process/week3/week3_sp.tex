\documentclass[en,hazy,blue,12pt,device = normal]{elegantnote}
\usepackage[breakable]{tcolorbox}
\usepackage{amsmath,amsthm,amssymb,bm,xcolor,bbm}
\usepackage{../spdefs}
\tcbset{breakable}
\pagecolor{white}

\title{Stochastic Process Exercise week 3}
\author{Tianlu Zhu \\ 2020571005}
% \institute{ShanghaiTech}
\date{}
\begin{document}
\maketitle
Chap 4: 1, 5, 13, 14, 16
\begin{enumerate}
    \item[1] Three white and three black balls are distributed in two urns in such a way that each contains three balls. We say that the system is in state \(i, i=0,1,2,3\), if the first urn contains \(i\) white balls. At each step, we draw one ball from each urn and place the ball drawn from the first urn into the second, and conversely with the ball from the second urn. Let \(X_n\) denote the state of the system after the \(n\)th step. Explain why \(\left\{X_n, n=0,1,2, \ldots\right\}\) is a Markov chain and calculate its transition probability matrix.
    
    \begin{tcolorbox}
        \sol 

        For each step, \(X_n\) given all information of two urns, and the choose process is depends on the information, having nothing to do with the past state, so this is a Markov process.

        The transition matrix is:
        \begin{align*}
            \textbf{P} = \begin{bmatrix}
                0&1&1&0 \\
                1/9&4/9&4/9&0 \\
                0&4/9&4/9&1/9\\
                0&0&1&0
            \end{bmatrix}
        \end{align*}
        With respect to \(\pi_n = (\mathbbm{1}_{X_n = 0},\mathbbm{1}_{X_n = 1},\mathbbm{1}_{X_n = 2},\mathbbm{1}_{X_n = 3},)\)
    \end{tcolorbox}

    \item[5] A Markov chain \(\left\{X_n:n\geq 0 \right\}\) with states \(0,1,2\) has the transition probability matrix
    \begin{align*}
        \textbf{P} = \begin{bmatrix}
            1/2 & 1/3 & 1/6 \\
            0& 1/3 & 2/3\\
            1/2 & 0 &1/2
        \end{bmatrix}
    \end{align*}
    If \(P\{X_0 = 0\}= P(X_0 = 1) = 1/4\), find \(\mean{X_3}\).

    \begin{tcolorbox}
        \sol

        Since that \(\pi_0 = (1/4,1/4,1/2)\),\[\pi_3 = \pi_0 \textbf{P}^3 \approx ({0.4, 0.2, 0.39})\]
        Then \(\mean{X_3} = 0.98\)
    \end{tcolorbox}

    \item[13]Let \(\mathbf{P}\) be the transition probability matrix of a Markov chain. Argue that if for some positive integer \(r\), \(\mathbf{P}^r\) has all positive entries, then so does \(\mathbf{P}^n\), for all integers \(n \geqslant r\).
    
    \begin{tcolorbox}
        \sol

        We first prove \(\mathbf{P}^{r+1}\) have all positive entries, and the \(\mathbf{P}^n\) can be induced.

        Notate the state space as \(\Omega\). Since that \(\mathbf{P}^r\) has all positive entries, \(p_{ij }^{(r)} >0\) for all \(i,j\in\Omega\),
        \[p_{ij }^{(r+1)} = \sum_{k \in \Omega} p_{ik}^{(r)} p_{kj }^{(1)} > 0\]

        Then \(\mathbf{P}^{r+1}\) have all positive entries, so do \(\mathbf{P}^{r+2}\),\(\mathbf{P}^{r+3} \cdots\) and \(\mathbf{P}^{n}\)
    \end{tcolorbox}

    \item[14] Specify the classes of the following Markov chains, and determine whether they are transient or recurrent:
    \begin{align*}
        &\mathbf{P}_1=\left\|\begin{array}{ccc}
        0 & \frac{1}{2} & \frac{1}{2} \\
        \frac{1}{2} & 0 & \frac{1}{2} \\
        \frac{1}{2} & \frac{1}{2} & 0
        \end{array}\right\|,
        &\mathbf{P}_2=\left\|\begin{array}{llll}
        0 & 0 & 0 & 1 \\
        0 & 0 & 0 & 1 \\
        \frac{1}{2} & \frac{1}{2} & 0 & 0 \\
        0 & 0 & 1 & 0
        \end{array}\right\|,
    \end{align*}
    \begin{align*}
        &\mathbf{P}_3=\left\|\begin{array}{lllll}
        \frac{1}{2} & 0 & \frac{1}{2} & 0 & 0 \\
        \frac{1}{4} & \frac{1}{2} & \frac{1}{4} & 0 & 0 \\
        \frac{1}{2} & 0 & \frac{1}{2} & 0 & 0 \\
        0 & 0 & 0 & \frac{1}{2} & \frac{1}{2} \\
        0 & 0 & 0 & \frac{1}{2} & \frac{1}{2}
        \end{array}\right\|,
        &\mathbf{P}_4=\left\|\begin{array}{ccccc}
        \frac{1}{4} & \frac{3}{4} & 0 & 0 & 0 \\
        \frac{1}{2} & \frac{1}{2} & 0 & 0 & 0 \\
        0 & 0 & 1 & 0 & 0 \\
        0 & 0 & \frac{1}{3} & \frac{2}{3} & 0 \\
        1 & 0 & 0 & 0 & 0
        \end{array}\right\|
    \end{align*}
    \begin{tcolorbox}
        \sol

        First we determine the classes. \({\bf P} _1\) has a class \(\{1,2,3\}\), \({\bf P_2}\) has a class \(\{1,2,3,4\}\), \({\bf P_3}\) have classes \(\{1,3\},\{2\},\{4,5\}\), \({\bf P_4}\) have classes \(\{1,2\},\{3\},\{4\},\{5\}\). 
        
        We apply the theorem: State \(i\) is recurrent if and only if \[\sumn p_{i i} ^{(n)} = \infty\]
        Also, state \(i\) is transient if and only if \[\sumn p_{i i} ^{(n)} = \frac{1}{1 - f_{ii}} < \infty\]

        For \({\bf P_1,P_2}\), they are irreducible and finite chain, so all the states are recurrent. For \({\bf P_3}\), all the classes are recurrent. Then consider \({\bf P_4}\), first two classes are recurrent, others are transient.
    \end{tcolorbox}

    \item[16] Show that if state \(i\) is recurrent and state \(i\) does not communicate with state \(j\), then \(P_{i j}=0\). This implies that once a process enters a recurrent class of states it can never leave that class. For this reason, a recurrent class is often referred to as a closed class.
    
    \begin{tcolorbox}
        \sol

        The communicated states with state \(i\) is named \(A\) and the others named \(B\). If \(P_{AB} \neq 0\), we have \(P_{BA} = 0\)
        \begin{align*}
            f_{ii } &= \sumn P_{AA}^n + P_{AB}^n  < 1
        \end{align*}
        Contradict to recurrent assumption. So \(P_{AB} = 0\). Let \(B = \{j \}\) that is \(P_{ij } = 0\)
    \end{tcolorbox}
\end{enumerate}


\end{document}